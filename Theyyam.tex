\documentclass[a4paper, 11pt]{article} % Font size (can be 10pt, 11pt or 12pt) and paper size (remove a4paper for US letter paper)

\usepackage[protrusion=true,expansion=true]{microtype} % Better typography
\usepackage{graphicx} % Required for including pictures
\usepackage{wrapfig} % Allows in-line images
\usepackage{enumitem} %%Enables control over enumerate and itemize environments
\usepackage{setspace}
\usepackage{amssymb, amsmath, mathrsfs} %%Math packages
\usepackage{stmaryrd}
\usepackage{mathtools}
\usepackage{mathpazo} % Use the Palatino font
\usepackage[T1]{fontenc} % Required for accented characters
\usepackage{array}
\usepackage{bibentry}
\usepackage[round]{natbib} %%Or change 'round' to 'square' for square backers
\setcitestyle{aysep={}}

\linespread{1.05} % Change line spacing here, Palatino benefits from a slight increase by default

%%% SANSKRIT DEFINITIONS %%%
\usepackage{semtrans}
\newcommand{\camunda}[0]{C\={a}mu\d{n}\d{d}\={a}}
\newcommand{\kavu}[0]{\textit{k\={a}v\^{u}}}
\newcommand{\kavus}[0]{\textit{k\={a}v\^{u}s}}
\newcommand{\Kavu}[0]{K\={a}v\^{u}}
\newcommand{\kanthakarnna}[0]{Ka\d{n}\d{t}h\={a}kar\d{n}\d{n}a}
\newcommand{\kanthakarnnan}[0]{Ka\d{n}\d{t}h\={a}kar\d{n}\d{n}an}
\newcommand{\kanthakarnnanre}[0]{Ka\d{n}\d{t}h\={a}kar\d{n}\d{n}an\U{r}e}
\newcommand{\puranic}[0]{Pur\={a}\d{n}ic}
\newcommand{\daiva}[0]{\textit{daiva}}
\newcommand{\kuttisastappanre}[0]{Ku\d{t}\d{t}i\'{s}\={a}stappan\U{r}e}
\newcommand{\vellattam}[0]{Vell\={a}\d{t}\d{t}a\.{m}}
\newcommand{\vasurimalavute}[0]{Vas\={u}rim\={a}layu\d{t}e}
\newcommand{\karincamundiyute}[0]{Kari\={n}c\={a}mu\d{n}\d{d}iyu\d{t}e}
\newcommand{\keattippatal}[0]{Ke\={a}\d{t}\d{t}ipp\={a}\d{t}al}
\newcommand{\gulikan}[0]{Gu\d{l}ikvan}
\newcommand{\tira}[0]{Ti\U{r}a}
\newcommand{\kaitacamundiyute}[0]{Kaitac\={a}mu\d{n}\d{d}iyu\d{t}e}
\newcommand{\visnumurtti}[0]{Vi\d{s}\d{n}um\={u}rtti}
\newcommand{\raktacamundi}[0]{Raktac\={a}mu\d{n}\d{d}i}
\newcommand{\meleri}[0]{M\={e}l\={e}ri}
\newcommand{\kayyelkkal}[0]{Kayy\={e}lkkal}
\newcommand{\muccileatt}[0]{muccil\={e}\={a}\d{t}\d{t}}
\newcommand{\bhagavatiyute}[0]{bhagavatiyu\d{t}e}
\newcommand{\tirumuti}[0]{tirumu\d{t}i}
\newcommand{\aratikkal}[0]{\={a}\U{r}\={a}\d{t}ikkal}
\newcommand{\kaliyatta}[0]{ka\d{l}iy\={a}\d{t}\d{t}a}





\newcommand{\corner}[1]{\ulcorner#1\urcorner} %%Corner quotes
\newcommand{\tuple}[1]{\langle#1\rangle} %%Angle brackets
\newcommand{\set}[1]{\lbrace#1\rbrace} %%Set brackets
\newcommand{\interpret}[1]{\llbracket#1\rrbracket} %%Double brackets
%\DeclarePairedDelimiter\ceil{\lceil}{\rceil}    

\makeatletter
\renewcommand\@biblabel[1]{\textbf{#1.}} % Change the square brackets for each bibliography item from '[1]' to '1.'
\renewcommand{\@listI}{\itemsep=0pt} % Reduce the space between items in the itemize and enumerate environments and the bibliography

\renewcommand{\maketitle}{ % Customize the title - do not edit title and author name here, see the TITLE block below
\begin{flushright} % Right align
{\LARGE\@title} % Increase the font size of the title

\vspace{10pt} % Some vertical space between the title and author name

{\@author} % Author name
\\\@date % Date

\vspace{30pt} % Some vertical space between the author block and abstract
\end{flushright}
}

%----------------------------------------------------------------------------------------
%	TITLE
%----------------------------------------------------------------------------------------

\title{\textbf{Theyyam}} % Subtitle

\author{\textit{Kerelian Adventures}} % Institution

\date{\today} % Date

%----------------------------------------------------------------------------------------

\begin{document}

\maketitle % Print the title section

\thispagestyle{empty}

%----------------------------------------------------------------------------------------

\section{Definition}
  \label{sec:Definition}

Theyyam (after Skt. \daiva\ "deity") is an elaborate tantric, tribal-Hindu and shamanic ritual, to have divinities descend through oracles in trance who get possessed by divine spirits, held in sacred groves called \kavus\ or temple courtyards in the northern Malabar region of south-west India consisting of districts in northern Kerala and southern Karnataka. It is performed mostly by low-caste Dalit androgynous males with hereditory ritual roles and predominantly involves theatrical performances with music, cos-play, chanting and dancing, including animal sacrifices for \camunda and fire ceremonies for \kanthakarnna. Theyyam practitioners go into shamanic trance often with the aid of alchoholic beverage made of fermented coconuts called \textit{toddy} offered to them. They also play vital socio-shamanic roles such as providing healings, blessings and psychological support in a given community hosting the ritual. Theyyam's ritual art forms across northern Malabar are highly diverse with presence of countlessly various deities derived from both Hindu \puranic and local non-Hindu mythologies. Although Theyyam is practised in Hindu context, its notable tribal and shamanic elements are considered to reflect its pre-Hindu or even pre-historic prototypes as seen in the theyyam of Gulykan, worshipped as Shiva, of tribal and shamanic nature, characteristically empolying tribal garments and headresses made of tender coconut leaves and oblong wodden shamanic masks.




\section{Program}%
  \label{sec:Program}

This years's Thiramahotsavam at Eranjoli, Chungam Sri Panakkadan Malayan Kav will be held on December 15, 16, 17, 18, 2022 (1198 Scorpio 29, 30 Sagittarius 1, 2), Thursday, Friday, Saturday and Sunday.

\subsection{Panakkatan Malayan \textit{K\={a}v\^{u}} at Eranhol, Chungam}%
  \label{sub:Panakkatan Malayan Kavu at Eranhol, Chungam}
  

Thiramahotsavam is held on 1st and 2nd Dhanu in the Malayalan calender, that fell 17th and 18th Dec 2022. 
We attended all the Theyyam performances on both days as listed below.

\subsubsection*{\em Day 1 (17th Dec 2022)}%
  \label{sub:Day 1 (17th Dec 2022)}
  
We missed Vellattam of Kandakarna but the proliminary deity was still present. 
Gulykan theyyam is worshiped as Lord Shiva.

\begin{enumerate}
\item Kuttishastappan's Vellattam (Vid 1: \kuttisastappanre\ \vellattam) 
  % (കുട്ടിശാസ്തപ്പന്റെ വെള്ളാട്ടം/Ku\d{t}\d{t}i\'{s}\={a}stappan\U{r}e \vellattam): Vid 1
\item Preliminary Kandakarna (Vid 2: Kandakarna \vellattam/\kanthakarnnanre\ \vellattam missed) 
  % (Kandakarna Vellattam/കണ്ഠാകർണ്ണന്റെ വെള്ളാട്ടം/Ka\d{n}\d{t}h\={a}kar\d{n}\d{n}an\U{r}e \vellattam missed): Vid 2
\item Vasoorimala's Vellattam (Vid 3: \vasurimalavute\ \vellattam) 
  % (വസൂരിമാലയുടെ വെള്ളാട്ടം രാത്രി/Vas\={u}rim\={a}layu\d{t}e \vellattam): Vid 3
\item Karimchamundi's Kottipadal (Vid 4: \karincamundiyute\ \keattippatal) 
  %(ന് കരിംചാമുണ്ഡിയുടെ കൊട്ടിപ്പാടൽ/Kari\={n}c\={a}mu\d{n}\d{d}iyu\d{t}e Ke\={a}\d{t}\d{t}ipp\={a}\d{t}al): Vid 4
\item Gulykan Thira (Vid 5 \& 6: \gulikan\ \tira) 
  % (ഗുളികൻ തിറ/Gu\d{l}ikan Ti\U{r}a): Vid 5 \& 6
\item Karimchamundi's Thira (Vid 7: \karincamundiyute\ Ti\U{r}a) 
  % (കരിംചാമുണ്ഡിയുടെ തിറ/Kari\={n}c\={a}mu\d{n}\d{d}iyu\d{t}e Ti\U{r}a): Vid 7
\end{enumerate}

\subsubsection*{\em Day 2 (16th Dec 2022)}%
  \label{sub:Day 2 (16th Dec 2022)}

We missed the first two \textit{thiras} of Kuttishastappan and Vasoorimala but made it to Kaitachamundi's Thira. 
Yet Kuttishastappan and Vasoorimala were still present in the \kavu. 
Vasoorimala is the goddess of pestilence.

\begin{enumerate}
\item Kandakarnan's Thira (Vid 8 \& 9: Ka\d{n}\d{t}h\={a}kar\d{n}\d{n}an Ti\U{r}a) 
  % (കണ്ഠാകർണ്ണൻ തിറ/ka\d{n}\d{t}h\={a}kar\d{n}\d{n}an ti\U{r}a): Vid 8 \& 9
\item Kuttishastappan (Vid 10: Kuttishastappan's Thira/\kuttisastappanre \tira missed) 
  % (Kuttishastappan's Thira/കുട്ടിശാസ്തപ്പന്റെ തിര/Ku\d{t}\d{t}i\'{s}\={a}stappan\U{r}e Tira missed): Vid 9
\item Vasoorimala (Vid 11: Vasoorimala's Thira/Vas\={u}rim\={a}layu\d{t}e \tira missed) 
  % (Vasoorimala's Thira/വസൂരിമാലയുടെ തിര/Vas\={u}rim\={a}layu\d{t}e Tira missed): Vid 11
\item Kaitachamundi's Thira (Vid 12 \& 13: \kaitacamundiyute\ \tira)
  %(കൈതചാമുണ്ഡിയുടെ തിര/Kaitac\={a}mu\d{n}\d{d}iyu\d{t}e Tira): Vid 12 \& 13
\item[$*$.] Plus Vid 14 of Andy, Ben and Ken receiving shamanic blessings from Theyyam deities 
\end{enumerate}


\subsection{Muchilot Bhavgavati \textit{K\={a}v\^{u}} at Nellunni Vattapo, Mattannoor}%
  \label{sub:Muchilot Bhavgavati Kavu at Nellunni Vattapo, Mattannoor}

The temple in Muchilot Bhavgavati \Kavu\ at Nellunni Vattapo came to prominance under the Kottayam dynasty in association with local Kalleri tribals. 
The principal deity of the temple is Muchilot Bhagawati, the clan deity of the Vanya community, who was brought from Karivallur, where she is locally worshipped as Muchilotamma in the Muchilot Kavu on 2nd and 3rd Makaram/Kumbham in the Malayalam calender. 
The theyyam is performed at the temple in the Muchilot Bhavgavati \Kavu\ at Nellunni Vattapois at an annual festival held on the 1st, 2nd and 3rd Dhanu in the Malayalam calender with attendence of devotees from the Vaniya community.

We attended the Day 3 of the 3-day festival on 17th, 18th and 19th Dec 2022 with the following Theyyam perfomances.

\begin{enumerate}
\item Vishnumurthy Theyyam (Vid 15: \visnumurtti) 
  % (വിഷ്ണുമൂർത്തി/Vi\d{s}\d{n}um\={u}rtti): Vid 15
\item Raktachamundi Theyyam (Vid 16: \raktacamundi) 
  % (രക്തചാമുണ്ഡി/Raktac\={a}mu\d{n}\d{d}i): Vid 16
\item Puliyoor Kali Theyyam (Vid 17 \& 18: Puliy\={u}rk\={a}\d{l}i Bhagavathy) with the Maleri Kayelkal (\meleri\ \kayyelkkal) fire/ash ritual featuring Vishnumurthy
  % (പുലിയൂർകാളി ഭഗവതി/Puliy\={u}rk\={a}\d{l}i Bhagavathy) with the Maleri Kayelkal (മേലേരി കയ്യേൽക്കൽ/m\={e}l\={e}ri kayy\={e}lkkal) fire/ash ritual featuring Vishnumurthy: Vids 17 \& 18
\item Sri Muchilot Bhagavathy's Thirumudi Aaratikal Kaliyatta or Theyyam (Vid 19: \muccileatt\ \bhagavatiyute\ \tirumuti\ \aratikkal\ \kaliyatta)
  % (ശ്രീ മുച്ചിലോട്ട് ഭഗവതിയുടെ തിരുമുടി ആറാടിക്കൽ കളിയാട്ട/\'{s}rī muccil\={e}\={a}\d{t}\d{t} bhagavatiyu\d{t}e tirumu\d{t}i \={a}\U{r}\={a}\d{t}ikkal ka\d{l}iy\={a}\d{t}\d{t}a): Vid 19
\item[$*$.] Plus Ken's visit to the traditional temple wood carver Narayan Acharya in Maloor
\end{enumerate}




\section{Description}%
  \label{sec:Description}
  
The form of Theyyam (thirra) practised in the Panakkatan Malayan \Kavu\ located in Eranholi of Kannur district in Kerala begins with ritual offerings and blessings where live deities in their preliminary forms sanctify the space as well as those attending the ceremony.
The deities then perform three velattams, invoking \camunda\ through drumming, dance, and anointing the different alters in the \kavu.

forms, after which a new cast of dancers dressed in young palm leaves make offerings and receive garlands while the first deities watch from some ways away.
After the dance is complete, preparations are made for the first sacrifice of a live rooster where \camunda\ presents herself, dancing while hiding her face before devouring a live rooster, drinking its blood.
Having finished her meal of rooster and puffed rice, the possessed dancer crawls away in a trance, occasionally collapsing and writhing on the ground. 
A new troop of drummers is called on, as another rooster sacrifice takes place before the final morning fire dance, circumambulating the kava.
The crowd then disperses to reconvene in the afternoon for a final rooster sacrifice and ritual performances, finishing in the early evening.




\section{Notes}%
  \label{sec:Notes}
  
\begin{itemize}
  \item Evening of the 17th at 5pm
  \begin{itemize}
    \item Kuttishastappan's velattam\\
      (three slow dancers in red)
    \item Kandakarna's velattam\\
      (four orange faced dancers)
    \item Vesurimala's velattam\\
      (coconut leaf hat, mask, and stilts)
  \end{itemize}
  \item Evening of the 17th at 11:30pm
  \begin{itemize}
    \item Karimchamundi's Kottipadal\\
      (chicken sacrifice)
  \end{itemize}
  \item Morning of the 18th at 5am
  \begin{itemize}
    \item Gulykan Thira\\
      (tall red coconut leaf hat dance)
    \item Thira of Karimchamundi\\
      (second chicken sacrifice)
  \end{itemize}
\end{itemize}















\vfill

\bibliographystyle{Phil_Review} %%bib style found in bst folder, in bibtex folder, in texmf folder.
\bibliography{Zotero} %%bib database found in bib folder, in bibtex folder


\end{document}
